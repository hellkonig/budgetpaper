\documentclass[useAMS,usenatbib]{mn2e}
\input psfig.sty
\usepackage{array}
%\usepackage{graphicx}
%\usepackage{epstopdf}

\voffset=-0.8in

\def \gas {\textsc{gasoline}}
\def \gastwo {\textsc{ESF-gasoline2}}
\def \changa {\textsc{changa}}
\def \mean#1{\left< #1 \right>}

%%MISC
\def \aj {AJ}
\def \apj {ApJ}
\def \apjl {ApJL}
\def \mnras {MNRAS}
\def \apjs {ApJS}
\def \aap {A\&A}
\def \nat {Nature}
\def \pasp {PASP}
\def \na {NewA}

\def \etal {et~al.~}
\def \eg{e.g.}
\def \Section{\S}
\def \spose#1{\hbox  to 0pt{#1\hss}}  
\def \lta{\mathrel{\spose{\lower 3pt\hbox{$\sim$}}\raise  2.0pt\hbox{$<$}}}
\def \gta{\mathrel{\spose{\lower  3pt\hbox{$\sim$}}\raise 2.0pt\hbox{$>$}}}
\def \ion#1#2{#1{\footnotesize{#2}}\relax}
\def \ha       {H$\alpha$}
\def \hi       {\ion{H}{I}}
\def \hii      {\ion{H}{II}}
\def \oii      {[\ion{O}{II}]}

%%UNITS
\def \kmsmpc {\>{\rm km}\,{\rm s}^{-1}\,{\rm Mpc}^{-1}}
\def \kms {\ifmmode  \,\rm km\,s^{-1} \else $\,\rm km\,s^{-1}  $ \fi }
\def \kpc {\ifmmode  {\rm kpc}  \else ${\rm  kpc}$ \fi  }  
\def \hkpc {\ifmmode  {h^{-1}\rm kpc}  \else ${h^{-1}\rm kpc}$ \fi  }  
\def \hMpc {\ifmmode  {h^{-1}\rm Mpc}  \else ${h^{-1}\rm Mpc}$ \fi  }  
\def \Msun {\ifmmode \rm M_{\odot} \else $\rm M_{\odot}$ \fi}
\def \hMsun {\ifmmode h^{-1}\,\rm M_{\odot} \else $h^{-1}\,\rm M_{\odot}$ \fi}
\def \hhMsun {\ifmmode h^{-2}\,\rm M_{\odot}\else $h^{-2}\,\rm M_{\odot}$ \fi}
\def \Lsun {\ifmmode L_{\odot} \else $L_{\odot}$ \fi} 
\def \hhLsun {\ifmmode h^{-2}\,\rm L_{\odot} \else $h^{-2}\,\rm L_{\odot}$ \fi}

%%COSMOLOGY

\def\LCDM{$\Lambda$CDM }
\def \LCDM {\ifmmode \Lambda{\rm CDM} \else $\Lambda{\rm CDM}$ \fi}
\def \sig8 {\ifmmode \sigma_8 \else $\sigma_8$ \fi} 
\def \Omegam {\ifmmode \Omega_{\rm m} \else $\Omega_{\rm m}$ \fi} 
\def \Omegab {\ifmmode \Omega_{\rm b} \else $\Omega_{\rm b}$ \fi} 
\def \Omegar {\ifmmode \Omega_{\rm r} \else $\Omega_{\rm r}$ \fi} 
\def \fbar {\ifmmode f_{\rm b} \else $f_{\rm b}$ \fi} 
\def \OmegaL {\ifmmode \Omega_{\rm \Lambda} \else $\Omega_{\rm \Lambda}$\fi} 
\def \Deltavir {\ifmmode \Delta_{\rm vir} \else $\Delta_{\rm vir}$ \fi}
\def \rhocrit {\ifmmode \rho_{\rm crit} \else $\rho_{\rm crit}$ \fi}

%DARK MATTER
\def \rs {\ifmmode r_{\rm s} \else $r_{\rm s}$ \fi} 
\def \rh {\ifmmode r_{\rm h} \else $r_{\rm h}$ \fi} 
\def \Rvir {\ifmmode R_{\rm vir} \else $R_{\rm vir}$ \fi}
\def \Vvir {\ifmmode V_{\rm  vir} \else  $V_{\rm vir}$  \fi} 
\def \Vmax {\ifmmode V_{\rm  max} \else  $V_{\rm max}$  \fi} 
\def \Mvir {\ifmmode M_{\rm  vir} \else $M_{\rm  vir}$ \fi}  
\def \Mhalo {\ifmmode M_{200} \else $M_{200}$ \fi}  
\def \Nvir {\ifmmode N_{\rm  vir} \else $N_{\rm  vir}$ \fi}  
\def \Jvir {\ifmmode J_{\rm vir} \else $J_{\rm vir}$ \fi} 
\def \Evir {\ifmmode E_{\rm vir} \else $E_{\rm vir}$ \fi} 
\def \lam {\ifmmode \lambda  \else $\lambda$ \fi} 
\def \lamp {\ifmmode \lambda^{\prime} \else $\lambda^{\prime}$  \fi} 
\def \lampc {\ifmmode \lambda^{\prime}_{\rm c} \else
  $\lambda^{\prime}_{\rm c}$  \fi} 

\def \xoff {\ifmmode x_{\rm off} \else $x_{\rm off}$ \fi}
\def \rhorms {\ifmmode \rho_{\rm rms} \else $\rho_{\rm rms}$ \fi}
\def \qbar {\ifmmode \bar{q} \else $\bar{q}$ \fi}

%%BARYONS
\def \Mb {\ifmmode M_{\rm b} \else $M_{\rm b}$ \fi} 
\def \eSF {\ifmmode \epsilon_{\rm SF} \else $\epsilon_{\rm SF}$ \fi} 
\def \Md {\ifmmode M_{\rm d} \else $M_{\rm d}$ \fi} 
\def \Mg {\ifmmode M_{\rm g} \else $M_{\rm g}$ \fi} 
\def \Rb {\ifmmode R_{\rm b} \else $R_{\rm b}$ \fi} 
\def \Rd {\ifmmode R_{\rm d} \else $R_{\rm d}$ \fi} 
\def \Rg {\ifmmode R_{\rm g} \else $R_{\rm g}$ \fi} 
\def \mgal {\ifmmode m_{\rm gal} \else $m_{\rm gal}$ \fi} 
\def \rj {\ifmmode {\cal R}_j \else ${\cal R}_j$ \fi} 
\def \lamgal {\ifmmode \lambda_{\rm gal} \else $\lambda_{\rm gal}$ \fi} 
\def \Vcirc {\ifmmode V_{\rm circ} \else $V_{\rm circ}$ \fi} 
\def \Vrot {\ifmmode V_{\rm rot} \else $V_{\rm rot}$ \fi} 
\def \Vflat {\ifmmode V_{\rm flat} \else $V_{\rm flat}$ \fi} 
\def \Mstar {\ifmmode M_{\rm star} \else $M_{\rm star}$ \fi} 
\def \Mgas {\ifmmode M_{\rm gas} \else $M_{\rm gas}$ \fi} 
\def \Mbar {\ifmmode M_{\rm bar} \else $M_{\rm bar}$ \fi}
\def \Rbar {\ifmmode R_{\rm bar} \else $R_{\rm bar}$ \fi} 

%%MASS-TO-LIGHT RATIOS
\def \DeltaIMF {\ifmmode \Delta_{\rm IMF} \else $\Delta_{\rm IMF}$ \fi}

\def \VV {\ifmmode V_{\rm 2.2}/V_{200} \else $V_{2.2}/V_{200}$ \fi} 
\def \dvr {\ifmmode \partial_{\rm VR} \else $\partial_{\rm VR}$ \fi} 

%%%%%%%%%%%%%%%%%%%%%%%%%%%%%%%%%%%%%%%%%%%%%%%%%%%%%%%%%%%%%%%%%%%%%%

\title[Baryon Budget] {NIHAO VII: Predictions for the galactic baryon budget in dwarf to Milky Way mass haloes}

\author[Wang et al.]{Liang Wang$^{1,3}$\thanks{wangliang@pmo.ac.cn}, Aaron A. Dutton$^{2,3}$,
  Gregory S. Stinson$^3$, Andrea V. Macci\`o$^{2,3}$, 
\newauthor{Thales Gutcke$^3$, Xi Kang$^1$}\\
$^1$Purple Mountain Observatory, the Partner Group of MPI f\"ur Astronomie, 2 West Beijing Road, Nanjing 210008, China\\
$^2$New York University Abu Dhabi, PO Box 129188, Abu Dhabi, UAE\\
$^3$Max-Planck-Institut f\"ur Astronomie, K\"onigstuhl 17, 69117 Heidelberg, Germany}
\begin{document}

\date{to be submitted to MNRAS}
             
\pagerange{\pageref{firstpage}--\pageref{lastpage}}\pubyear{2015}

\maketitle           

\label{firstpage}
             
%%%%%%%%%%%%%%%%%%%%%%%%%%%%%%%%%%%%%%%%%%%%%%%%%%%%%%%%%%%%%%%%%%%%%%

\begin{abstract}
  We use the NIHAO galaxy formation simulations to make predictions
  for the baryonic budget  in present day galaxies ranging from dwarf
  ($\Mhalo\sim10^{10} \Msun$) to Milky Way ($\Mhalo\sim10^{12} \Msun$)
  masses.  The sample is made of 88 independent high resolution
  cosmological zoom-in simulations.  NIHAO galaxies reproduce key
  properties of observed galaxies, such as the stellar mass vs halo
  mass and cold gas vs stellar mass relations. Thus they make
  plausible predictions for the baryon budget.  We present the mass
  fractions of stars, cold gas ($T<10^4$K), cool gas ($10^4 < T <
  10^5$K), warm-hot gas ($10^5 < T < 10^7$K), and hot gas ($10^7$K$ <
  T$), inside the virial radius, $R_{200}$.  Compared to the predicted
  baryon mass, using the dark halo mass and the universal baryon
  fraction, $f_{\rm b}\equiv \Omega_{\rm b}/\Omega_{\rm m}=0.15$, we
  find that all of our haloes {\bf are missing baryons. The missing
    mass been relocated past 2 virial radii, and is dominated by cool gas.}
   Haloes of mass $\Mhalo\sim 10^{10}\Msun$ are missing
  $\sim 90\%$ of their baryons.  More massive haloes ($\Mhalo\sim
  10^{12}\Msun$) retain a higher fraction of their baryons, with $\sim
  30\%$ missing, consistent with observational estimates.
  
%  Finally we show that the baryonic budget of most massive 
%  galaxies in NIHAO suite can match the observation for most 
%  components but the cool gas, which requires further 
%  theoretical and observational work to find the answer where
%  the cool gas actually is.
\end{abstract}

\begin{keywords}
  galaxies: evolution -- galaxies: formation -- galaxies: dwarf -- galaxies: spiral -- 
  methods: numerical -- cosmology: theory
\end{keywords}

\setcounter{footnote}{1}

%%%%%%%%%%%%%%%%%%%%%%%%%%%%%%%%%%%%%%%%%%%%%%%%%%%%%%%%%%%%%%%%%%%%%%
%% SECTION 1: INTRODUCTION
%%%%%%%%%%%%%%%%%%%%%%%%%%%%%%%%%%%%%%%%%%%%%%%%%%%%%%%%%%%%%%%%%%%%%%

\section{Introduction}
\label{sec:intro}
Cosmic structure formation has redistributed the baryons from  a
nearly uniformly distributed plasma into a variety of states,
including stars, stellar remnants, cold (atomic and molecular) gas,
and hot (ionized) gas. The theories of galaxy formation can predict
the amount of mass in these different states, which can in turn be tested by
observational constraints.  

%% missing baryon problem
On cosmological scales, the ratio between the total baryonic and
gravitating mass is measured to be $f_{\rm b}\equiv \Omega_{\rm
  b}/\Omega_{\rm m}\simeq 0.15$ (The Planck Collaboration 2014).
However, the cold baryonic mass density implied by several galaxy baryon
estimates is only 3-8\% of the big bang nucleosynthesis expectation
\citep{Persic92, Fukugita98,  Bell03, McGaugh10}.  The majority of the
cosmic baryons are thought to be in the form of hot gas around or
between galaxies \citep{Cen09}. Until recently only a fraction of
these baryons had been detected \citep{Bregman07, Shull12}.  This
discrepancy is referred to as the ``missing baryon problem''.
{\bf Several theoretical studies with cosmological simulations
have constrained the phase and cosmological environments of
the potential reservoirs of the missing baryons 
\citep{Yoshida05, He05, Dave10, Zhu11, Haider16}.}

{\bf The circum galactic medium (CGM) is a major potential  reservoir
  of the missing baryons.}  Recent advances in the detection of  gas
in the CGM have come  from the COS survey \citep{Tumlinson11,
  Tumlinson13, Thom12,  Werk12, Werk13}.  On the scale of Milky Way
mass haloes $\Mhalo \sim 10^{12}\Msun$ a significant amount of warm
($10^4 < T < 10^7$K) gas has been detected \citep{Werk14}, accounting
for 33-88\% of the baryon budget. In the future such observations will
be extended to a wider range of halo masses.
%{\bf ??? To understand
%  the physical properties of the CGM and the origins of components in
%  each phase has been shown to be able to test the impact of feedback
%  models \citep{Sharma12,Marasco13}???
  {\bf The physical properties of the CGM has been shown to be able to
    test feedback models \citep{Sharma12,Marasco13}.  \citet{Dave09}
    predicted galactic halo baryon fractions of galaxies with halo
    masses ranging from $10^{11} \Msun$ to $10^{13} \Msun$ using
    cosmological hydrodynamical simulations with a well-constrained
    model for galactic outflows.  They found that, without the outflow
    model, the baryon fraction inside the virial radius is roughly the
    cosmic baryonic fraction, but with the outflow model, the baryon
    fraction is increasingly suppressed in lower mass haloes.  By
    comparing results at $z=3$ and $z=0$, they showed that large
    haloes remove their baryons at early times while small haloes lose
    baryons more recently due to the wind material taking longer to
    return to low-mass galaxies than high-mass galaxies.
%  
  \citet{Sokolowska16} studied the halo gas of three Milky way-sized
  galaxies using cosmological zoom-in simulations. They found that
  most of missing baryons actually resides in warm-hot and hot gas
  which contribute to 80\% of the total gas reservoir.  The recovered
  baryon fraction within 3 virial radii is 90\%.  The warm-hot medium
  is sensitive to the feedback model so that a reliable spatial
  mapping of the warm-hot medium will provide  a stringent test for
  feedback models.}


In this paper we make predictions for the baryonic budget for stars,
cold, warm and hot gas in and around the virial radius of haloes of
mass ranging from $\Mhalo\sim 10^{10}\Msun$ to $10^{12}\Msun$. We use
a sample of 88 zoom-in galaxy formation simulations from the NIHAO
project. NIHAO galaxies are consistent with the stellar mass vs halo
mass relations from halo abundance matching since redshift $z\sim 4$
\citep{Wang15}, the galaxy star formation rate vs stellar mass
relation since $z\sim 4$ \citep{Wang15}, and the cold gas mass vs
stellar mass relation at $z\sim 0$ \citep{Stinson15}.  Therefore,
the simulations make plausible predictions for the mass fractions and
physical locations of the warm and hot gas components.  We find that
all the haloes contain less baryons than expected according to the
cosmic baryonic fraction, but the missing fraction is strongly mass
dependent.  {\bf Mention Gutcke et al. 2016}

This paper is organized as follows: The cosmological hydrodynamical
simulations including star formation and feedback are briefly
described in  \S\ref{sec:sims}; In \S\ref{sec:budget} we present the
results including the baryonic budget, baryon distribution, and a
comparison with observations; \S\ref{sec:sum} gives a summary of our
results.

%%%%%%%%%%%%%%%%%%%%%%%%%%%%%%%%%%%%%%%%%%%%%%%%%%%



%%%%%%%%%%%%%%%%%%%%%%%%%%%%%%%%%%%%%%%%%%%%%%%%%%%
%% SECTION 2 SIMULATIONS
%%%%%%%%%%%%%%%%%%%%%%%%%%%%%%%%%%%%%%%%%%%%%%%%%%%
\section{Simulations} 
\label{sec:sims}

In this study we use simulations from the NIHAO (Numerical
Investigation of a Hundred Astrophysical Objects) project \citep{Wang15}.  
The initial conditions are created to keep the same numerical
resolution across the whole mass range with typically a million dark
matter particles inside the virial radius of the target halo at 
redshift $z=0$.  The halos to be re-simulated at higher resolution
with baryons have been extracted from 3 different pure N-body
simulations with a box size of 60, 20 and 15 $h^{-1}$ Mpc
respectively.  We adopted the  latest compilation of cosmological
parameters from the Planck  satellite \citep{Planck14}. 
More information on the collisionless parent simulations and
sample  selection can be found in \citet{Dutton14} and \citet{Wang15}

We use the SPH hydrodynamics code {\sc gasoline} \citep{Wadsley04},
with a revised treatment of  hydrodynamics as described in
\citet{Keller14}.  The code includes a subgrid model for turbulent
mixing of metal and energy \citep{Wadsley08}, heating and cooling
include photoelectric heating of dust grains, ultraviolet (UV) heating
and ionization and  cooling due to hydrogen, helium and metals
\citep{Shen10}.  The star formation and feedback modeling follows what
was used in the MaGICC simulations \citep{Stinson13}.  There are two
small changes in NIHAO simulations: The change in  number of neighbors
and the new combination of softening length and  particle mass increases
the threshold for star formation from  9.3 to 10.3
cm$^{-3}$, the increase of pre-SN feedback efficiency $\epsilon_{\rm
  ESF}$, from 0.1 to 0.13.  The more detail on star formation and
feedback modeling can be found in \citet{Wang15}.



%%%%%%%%%%%%%%%%%%%%%%%%%%%%%%%%%%%%%%%%%%%%%%%%%%%%%%%%%%%%%%%%%%%%%%%%%%%%
%% SECTION 3 RESULTS
%%%%%%%%%%%%%%%%%%%%%%%%%%%%%%%%%%%%%%%%%%%%%%%%%%%%%%%%%%%%%%%%%%%%%%%%%%%%%%%
%\section{Results}
%\label{sec:results}

%We study the baryonic budget using the full set simulations of the NIHAO project.  
%The large range of the mass covered by our simulations enables
%us see the baryon fractions and mass profiles of different 
%galaxies, depending on the halo mass and stellar content. 

%% FIGURE 1
\begin{figure}
\centerline{
%  \psfig{figure=barbudget_linear_v3.eps,width=0.5\textwidth}}
%\centerline{
%  \psfig{figure=barbudget_v3.eps,width=0.5\textwidth}
  \psfig{figure=barbudget_linearlog.eps,width=0.5\textwidth}
}
\caption{Fractional baryon content of our NIHAO simulations  as a
  function of halo mass. The green points show the ratio between the
  baryonic mass (stars + gas) inside the virial radius and the total
  baryonic mass associated with the dark matter halo. The blue points
  show the corresponding fraction for the stars. The solid green line
  and shaded region shows a double power-law fit, together with the
  1$\sigma$ scatter. For the stellar mass fraction we show several
  relations from halo abundance matching.    The linear (upper panel)
  and logarithmic (lower panel) scales emphasize the large amount of
  ``missing'' baryons and the low star formation efficiencies.}
\label{fig:budget}
\end{figure}

%% FIGURE 2
\begin{figure*}
\centerline{
  \psfig{figure=budgetpro.eps,width=0.99\textwidth}
}
\caption{Radial profile of the mass fraction of the gas in each phase
         to total baryon mass in each radial bin 
         at z=0 for all galaxies in NIHAO sample.
         Each solid line is from one galaxy and color coded with
         the halo mass. {\bf tickmarks overlap. fix. use $R_{200}$ for virial radius}}
\label{fig:corona}
\end{figure*}




\section{Baryon budget}
\label{sec:budget}

We define the fiducial baryonic mass as:
\begin{equation}
M_{\rm b} \equiv M_{\rm b}(R_{200})= \frac{f_{\rm b}}{1-f_{\rm b}}M_{\rm dm}(R_{200}) 
\label{equ:mb}
\end{equation}
where the $M_{\rm dm}$ is the total dark matter mass in the halo, and
the $f_{\rm b} = \Omegab/\Omegam \sim 0.15$ is the cosmic baryon
fraction (the ratio between baryon density and mass density including baryonic mass
plus dark matter), so that the  $M_{\rm b}$ is the baryonic mass inside the
virial radius if the baryons follow the dark matter closely.

Fig.~\ref{fig:budget} shows the ratio between the  mass of each baryon
component inside the virial radius  to the fiducial baryonic mass for
the most massive galaxy in each zoom-in region. We present the
fractions of total stellar mass (blue points), and the total baryonic
mass including stellar mass plus gas mass (green points).  For the
stellar mass fraction we also show the relations from the halo
abundance  matching \citep{Moster13, Behroozi13, Kravtsov14}.  The
grey area is the one sigma scatter around the mean value  from
\citet{Kravtsov14}.

We tried to capture the behavior of the baryonic mass fraction as a 
function of the halo mass using a double power law formula:
\begin{equation}
\frac{f}{f_0} = \left( 
                        \frac{M_{200}}{\mathcal{M}_0} 
                        \right)^\alpha 
                        \left\{ 0.5 \left[ 1+\left( 
                        \frac{M_{200}}{\mathcal{M}_0} 
                        \right)^\gamma \right] 
                        \right\}^{\frac{\beta-\alpha}{\gamma}}.
\end{equation}
In this formula, the lower and higher mass ends have logarithmic slope
$\alpha$ and $\beta$, respectively, while $\gamma$ regulates how 
sharp the transition is from the lower to the higher ends.
The best fit parameters are as follows:
\begin{eqnarray}
\mathcal{M}_0  &=&  6.76 \times 10^{10} \nonumber \\
f_0  &=&   0.336 \nonumber \\
\alpha  &=&   0.684 \\
\beta  &=&   0.205 \nonumber \\
\gamma  &=&   3.40\nonumber  
\end{eqnarray}
The green shaded region indicates the scatter about the best fit line,
{\bf which is 0.151 for haloes with mass in the range of  
$3\times 10^{9}\Msun < M_{200} <2\times 10^{10} \Msun$, 0.236 for
halo mass in 
$2\times 10^{10}\Msun < M_{200} < 7\times 10^{10} \Msun$,
0.125 for halo mass in
$7\times 10^{10}\Msun < M_{200} < 3\times 10^{11} \Msun$
and 0.0518 for halo mass in
$3\times 10^{11}\Msun < M_{200} < 3.5\times 10^{12} \Msun$.}


The trends of each component fraction are similar, in that  the
fractions are relatively low in low mass haloes, and increase as the
halo mass increases.  The main difference between the different
components is the slope, with the baryonic mass fraction having a
shallower slope than the stellar mass fraction.  This is because in
low mass haloes ($\Mhalo\sim 10^{10}\Msun$) most of the baryons are in
the form of gas, while in the highest mass haloes we study
($\Mhalo\sim 10^{12}\Msun$) there are roughly equal amounts of stars
and gas.

Fig.~\ref{fig:budget} shows that all haloes in our study contain less
than the universal fraction of baryons. The upper panel uses a linear
y-axis scale, which highlights the large amount of baryons that are
missing, especially in low mass haloes. The logarithmic scale in the
lower panel highlights the power-law nature of the relations.

Since the haloes we study are above the mass where the cosmic UV
background prevents gas from cooling, the missing baryons have most
likely been ejected from the central galaxies in supernova/stellar
feedback driven winds.  Although the lower mass galaxies have
converted a smaller fraction of their available baryons into stars,
and hence there is proportionally less energy available to drive an
outflow, they have expelled a larger fraction of their baryons,
consistent with expectations from energy driven gas outflows
\citep[e.g.,][]{Dutton12}.



\subsection{Mass budget of the corona}
\label{sec:corona}

{\bf In Fig.~\ref{fig:corona}, we present the radial distribution of
  gas in different phases at $z=0$, normalized to the total baryon
  mass profile, such that for a given halo the four phases add up to
  unity.  All simulations share a common attribute.  The cold gas (T$<
  10^4$ K) is mostly located  near the center  ($R < 0.2 R_{200}$)
  where most stars in galaxies form.  In contrast, the cool ($10^4$K
  $<$ T $<$ $10^5$K) and  warm-hot ($10^5$K $<$ T $<$ $10^7$K) gas are
  located at large distances with roughly constant fractions up to 2
  times $R_{200}$.  The hot gas (T$>10^7$K) is a minority component
  for all galaxies in the NIHAO sample, with the maximum hot gas
  fraction at any radius being  less than 5\%.}

{\bf Despite these similarities, we find a considerably higher proportion
  of cool gas in lower mass galaxies (M$_{\rm tot} < 10^{11} \Msun$)
  in the whole corona region.  For higher mass galaxies, warm-hot gas
  dominates the corona  which signals higher efficiency  of feedback.
  Even beyond the virial radius, the cool and warm-hot gas has similar
  features as the gas within virial radius which reveals the gas
  surrounding galaxies within large distance is the major reservoir of
  baryons.}

%% FIGURE 3
\begin{figure}
\centerline{
  \psfig{figure=ratioprofile.eps,width=0.5\textwidth}
}
\caption{Baryon distribution of each galaxy from NIHAO simulations. 
 The lines are color coded by their halo mass, which shows a
  clear trend that the more massive haloes preserve more baryons at
  all radii.}
\label{fig:rps}
\end{figure}


%% FIGURE 4
\begin{figure}
\centerline{
  \psfig{figure=missing_vr.eps,width=0.5\textwidth}
}
\caption{Baryon radius as function of total virial mass.
         The points are color coded by the ratio 
         between baryon radius and the virial radius of each galaxy.}
\label{fig:missvr}
\end{figure}

%% FIGURE 5
\begin{figure*}
\centerline{
  \psfig{figure=budget_inout.eps,width=0.99\textwidth}
}
\caption{Mass fraction of gas in four phases (relative to the fiducial
  baryonic mass within virial radius). The blue and red points are for
  gas  inside and outside the virial radius, respectively. Cool gas is
  the dominant component of the CGM in NIHAO simulations.}
\label{fig:inout}
\end{figure*}





\subsection{Where are the missing baryons?}
\label{sec:where}

Fig.~\ref{fig:rps} shows the mass ratio profiles of total baryons  for
each simulation. Here the y-axis is the ratio between the baryonic to
dark matter mass, $M_{\rm b}(<R) / M_{\rm dm}(<R)$, enclosed
within a sphere of radius, $R$, normalized by the cosmic
baryon-to-dark matter ratio, $\Omega_{\rm b}/\Omega_{\rm dm}$.

Each solid curve represents a halo, and the curves are colored by
their halo mass (red for high masses to blue for low masses).  Broadly
speaking, the curves have a similar shape, with a normalization that
depends on halo mass. They have a cusp in  the central region where the
stars and cold gas dominate, then become flat in the outer region.
More massive haloes have higher baryon fractions at all radii.  At
small radii, the baryon to dark matter ratio is higher than the cosmic
value due to gas dissipation.  Even beyond the virial radius, there is
little change in the baryon fraction up to 2 virial radii.  We thus
conclude that the missing baryons are well outside of the virial
radius.

%The curves show a gradual transition based on mass that confirms 
%what we find in Fig.~\ref{fig:budget}. 
%In general terms, the lowest mass halos have a small peak but drop
%relatively steeply until $\sim$ 0.2R$_{200}$, then the curves rise slowly.
%As the masses increase, the central peaks rise and the curves in the 
%outer part are flater.
%Finally, the most massive galaxies have an almost constant curve 
%which are close but still less than 1.

{\bf To estimate how far the baryons escape, we measured the radius,
  $\Rbar$, within which the total baryon mass equals to the fiducial
  baryonic mass defined by Eq.~\ref{equ:mb}. This is a lower limit to
  the true extent of the missing baryons since the baryon mass
  includes gas and stars that belong to nearby lower mass haloes.
  Fig.~\ref{fig:missvr} shows the baryon radius of each galaxy as
  function of the virial mass.  We find the baryon radius generally
  increases with virial mass. But when normalized by the virial
  radius, the distance baryons are ejected to span a similar range of radii
  $\Rbar/R_{200}\sim 2-6$.}

{\bf In Fig.~\ref{fig:inout}, we calculate the fractions of gas (in each phase)
inside the virial radius (blue points) and between 
the virial and baryon radii (red points).
%
The upper-left panel shows the  fractions of cold gas. The fractions
of cold gas between the virial radius and baryon radius of galaxies with
halo masses in the range between $10^{9} \Msun$ to  $10^{10} \Msun$
are greater than 80\%. As the halo mass increases, this fraction goes
down dramatically, in the mass range between $10^{10} \Msun$ and
$10^{11} \Msun$ the fractions are less than  20\% for most of galaxies
and the fractions for galaxies with halo mass above $10^{11} \Msun$ are
almost 0.  For the cold gas inside the virial radius, the fractions
increase gradually as halo mass increases and the maximum fraction is
about 20\% with at a halo mass of $10^{12} \Msun$.
%
The upper-right panel shows the fractions of cool gas.  The fractions
of cool gas in region outside of virial radius  increase from 0 at
halo masses are $10^9 \Msun$ to 80\% at halo  masses are around
$10^{10} \Msun$ to $10^{11} \Msun$, then  decrease in higher mass
range and reach 0 again at halo masses are $10^{12} \Msun$. The cool
gas inside virial radius shares  a similar trend but the maximum
fraction is about 40\% at  halo masses are $10^{11} \Msun$.
%
The fractions for warm-hot gas are shown in the lower-left panel.
The trends of the gas inside and outside gas are quite similar
which increase as the halo masses increase monotonically and whose 
maximum values are only 20\%.
%
The hot gas shown in the lower-right panel and is negligible both
inside and outside the virial radius across the whole mass range we
study.
%
We thus conclude that, for galaxies with halo mass 
below $10^{11}$, the majority of baryons associated with the
dark matter halo are in the cold and cool phases, 
and are located well outside of virial radius. For the gas inside
virial radius, the majority is cool gas for the galaxies with
halo masses are $10^{11} \Msun$ and when halo masses are higher
than $10^{11} \Msun$ the fractions of cold gas, cool gas and 
warm-hot gas are comparable.}

%% FIGURE 6
\begin{figure}
\centerline{
  \psfig{figure=budget.eps,width=0.5\textwidth}
}
\caption{Baryonic budget 
  of NIHAO haloes of mass $3.5\times 10^{11} < M_{200}/\Msun <
  3.5 \times 10^{12}$ (blue points 
  with 1$\sigma$ error bars) compared with observations of 
  $M_{200} \sim 10^{12}\Msun$ haloes (shaded regions)
  }
\label{fig:comparison}
\end{figure}



%% TABLE 1
\begin{table*}
  \caption{The baryonic budget parameters for NIHAO galaxies
    in different halo  mass bins. We refer to gas in the temperature range  T
  $<$ $10^4$ K as cold; $10^4$ K $\leqslant$ T $<$ $10^5$ K as cool;
  $10^5$ K $\leqslant$ T $<$ $10^7$ K as warm;  and T $\geqslant$
  $10^7$ K as hot.}
\begin{center}
\begin{tabular}{ccccc}
\hline
\input{budgettable4.txt}
\hline
\end{tabular}
\label{tab:comparison}
\end{center}
\end{table*}



\subsection{Comparison with Observations of Milky Way mass haloes}
Since the CGM is too diffuse to create emission lines, it must be
observed using quasar absorption lines.  The COS-HALOs survey is
filling in details about the $z \sim 0$ CGM \citep{Peeples14,
  Tumlinson11, Tumlinson13, Werk12, Werk13, Werk14}.  For the CGM of
low-redshift $\sim L^*$ galaxies ($\Mstar\sim 10^{10.5}\Msun$),
\citet{Tumlinson13} and \citet{Peeples14} constrain the mass of  the
warm-hot CGM ($T \sim 10^{5-7}$K), \citet{Werk14} provides a strict
lower limit to the mass of cool material ($T \sim 10^{4-5}$K) in the
CGM of these galaxies.  In a study using X-rays,  \citet{Anderson13}
place a constraints on the mass of  hot gas ($T > 10^7$K) residing in
the extended hot halos.  

In Fig.~\ref{fig:comparison}, we show the mean values and standard deviation
of the mass fraction of stars and different components of gas in our
most massive galaxies ($3.49 \times 10^{11} \Msun$ $<$ M$_{200}$ $<$ 
$3.53 \times 10^{12} \Msun$) with blue points and error bars.
The gas is assigned to a range of temperature bins: 
cold gas (T $< 10^4$ K), cool gas ($10^4$ K $<$ T $< 10^5$ K), 
warm gas ($10^5$ K $<$ T $< 10^7$ K) and hot gas (T $> 10^7$ K).
The observational constraints are shown with the same colour scheme
in Fig. 11 in \citet{Werk14}.

In this plot \citet{Werk14} provides observational constraints for
CGM gas mass that are shown as the shaded bars.  The stellar  mass
range comes from halo abundance matching as described in
\citet{Kravtsov14}.   The cold disk gas mass comes from from
\citet{Dutton11}.  

The observations and the simulations match well in every phase {\bf
  except} the cool CGM gas, where the observations find $3\times$ the
mass that simulations predict.  If the observations are correct, the
simulations have either ejected {\bf cool gas too far}, or  they have
created a CGM with the wrong mix of gas temperatures. The total gas
fractions (0.39 in COS-HALOs, 0.41 in NIHAO) suggest the latter
option.   {\bf However, \citet{Stern16} developed a new method to
  constrain the physical conditions in the cool CGM from measurements
  of ionic  columns densities. This new method combines the
  information available from different sightlines during the photoionization
  modeling, and was applied to the COS-HALOs data,  yielding a total
  cool CGM mass within the virial radius of $1.3\times10^{10}\Msun$
  which is shown by the green hashed bar in Fig~\ref{fig:budget} and
  is in good agreement with our prediction.  As the
  Fig.~\ref{fig:corona} and Fig.~\ref{fig:inout} show, the cool gas is
  the most important component in CGM so that the more accurate
  knowledge of the physical properties of CGM are necessary to better
  understand the role of the CGM in galaxy formation.}
%
As the CGM of lower mass galaxies will soon be observed,
Table~\ref{tab:comparison} lists information about CGM
mass fractions of the different components of gas in haloes down to a
halo mass of $\sim 10^{10}\Msun$.




%%%%%%%%%%%%%%%%%%%%%%%%%%%%%%%%%%%%%%%%%%%%%%%%%%%%%%%%%%%%%%%%%%%%%%
%% SECTION 4: SUMMARY
%%%%%%%%%%%%%%%%%%%%%%%%%%%%%%%%%%%%%%%%%%%%%%%%%%%%%%%%%%%%%%%%%%%%%%
\section{Summary}
\label{sec:sum}

We have used the NIHAO galaxy simulation suite \citep{Wang15} to study
the statistical features of the baryonic budget and distribution
spanning halo masses of $\sim 10^{10}$ to $\sim 10^{12}\Msun$. NIHAO
is a large (currently 88) set of high resolution cosmological
hydrodynamical galaxy formation simulations. As shown in previous papers
the NIHAO galaxies reproduce several key observed scaling relations.
We summarize our results as follows:

\begin{itemize}
\item All of the NIHAO haloes have a lower baryon to dark matter ratio, 
      inside the virial radius, than the cosmic baryon fraction. 

\item {\bf Cold gas ($T<10^4$K) is mostly restricted to be within
      0.2 virial radii.  The cool gas ($10^4 < T < 10^5$K) dominates
      the corona at low masses ($M_{200}\lta 3\times 10^{11} \Msun$)
      while the warm-hot gas ($10^5 < T < 10^7$K) dominates at high
      masses ($M_{200}\gta 3\times 10^{11} \Msun$).}

\item The missing baryons in all haloes are beyond 2 virial radii.

\item {\bf Cool gas is a major component of the total baryons within
      R$_{\rm bar}$, most of which is outside of the virial radius.}

\item Lower mass haloes have lost a larger fraction of their baryons, 
      even though they convert a lower fraction of the baryons into stars.

\item For the highest mass haloes in our study $\sim 10^{12}\Msun$ our
      simulations are consistent with the observed  fractions of stars,
      cold gas, warm and hot gas.

\item {\bf For the cool gas we predict $f_{\rm cool}=0.11\pm0.06$
      which is significantly lower than the observations from COS-HALOs
      ($f_{\rm cool}=0.28-0.48$), but is in excellent agreement with the
      analysis of \citet{Stern16}.}

\end{itemize}


\section*{Acknowledgments} 

{\sc Gasoline} was written by Tom Quinn and James Wadsley. Without
their contribution, this paper would have been impossible.
%
The simulations were performed on the {\sc theo} cluster of the
Max-Planck-Institut f\"ur Astronomie and the {\sc hydra} cluster at
the Rechenzentrum in Garching; and the Milky Way supercomputer, funded
by the Deutsche Forschungsgemeinschaft (DFG) through Collaborative
Research Center (SFB 881) "The Milky Way System" (subproject Z2),
hosted and co-funded by the J\"ulich Supercomputing Center (JSC). We
greatly appreciate the contributions of all these computing
allocations.
%
AAD, GSS and AVM acknowledge support through the
Sonderforschungsbereich SFB 881 “The Milky Way System” (subproject A1)
of the German Research Foundation (DFG).  The analysis made use of the
pynbody package \citep{Pontzen13}.
%
The authors acknowledge support from the MPG-CAS through the
partnership programme between the MPIA group lead by AVM and the PMO
group lead by XK.
%
LW acknowledges support of the MPG-CAS student programme.
%
XK acknowledge the support from 973 program (No. 2015CB857003,
2013CB834900), NSFC project No.11333008 and the ``Strategic Priority
Research Program the Emergence of Cosmological Structures'' of the
CAS(No.XD09010000).


%%%%%%%%%%%%%%%%%%%%%%%%%%%%%%%%%%%%%%%%%%%%%%%%%%%%%%%%%%%%%%%%%%%%%%
%%  REFERENCES
%%%%%%%%%%%%%%%%%%%%%%%%%%%%%%%%%%%%%%%%%%%%%%%%%%%%%%%%%%%%%%%%%%%%%% 

\begin{thebibliography}{}

%%AAAAAAAA

% Extended Hot Halos around Isolated Galaxies Observed in the ROSAT All-Sky Survey
\bibitem[Anderson et al.(2013)]{Anderson13} Anderson, M.~E., Bregman, J.~N., Dai, X.\ 2013, \apj, 762, 106

% Fundamental differences between SPH and grid methods
\bibitem[Agertz et al.(2007)]{Agertz07} Agertz, O., Moore, B., Stadel, J.\ 2007, \mnras, 380, 963

%%BBBBBBBB

%The Average Star Formation Histories of Galaxies in Dark Matter Halos
% from z = 0-8
\bibitem[Behroozi et al.(2013)]{Behroozi13} Behroozi, P.~S.,
  Wechsler, R.~H., \& Conroy, C.\ 2013, \apj, 770, 57

%A First Estimate of the Baryonic Mass Function of Galaxies
\bibitem[Bell et al.(2003)]{Bell03} Bell, E.~F., McIntosh, D.~H., Katz, N., Weinberg, M.~D.,\ 2003, \apj, 585, 117

% The Search for the Missing Baryons at Low Redshift
\bibitem[Bregman (2007)]{Bregman07}
Bregman, J.~N.\ 2007, ARAA, 45, 221

%% CCCCCCCC

%Where are the baryons
\bibitem[Cen \& Ostriker (1999)]{Cen09} 
Cen, R.~Y., Ostriker, J.~P.\ 1999, \apj, 514, 1


%% DDDDDDDD

% Missing Halo Baryons and Galactic Outflows
\bibitem[Dav{\'e} (2009)]{Dave09} Dav{\'e}, R.\ 2009, ASPC, 419, 347D

% The intergalactic medium over the last 10 billion years - I. Lya absorption and physical conditions
\bibitem[Dav{\'e} et al.(2010)]{Dave10} Dav{\'e}, R., Oppenheimer, B.~D., Katz, N., et al.\ 2010, \mnras, 408, 2051

%Dark halo response and the stellar initial mass function in early-type and late-type galaxies
\bibitem[Dutton et al.(2011)]{Dutton11} Dutton, A.~A., Conroy, 
  C., van den Bosch, F.~C., et al.\ 2011, \mnras, 416, 322
  
%The baryonic Tully-Fisher relation and galactic outflows
\bibitem[Dutton(2012)]{Dutton12} Dutton, A.~A.\ 2012,
  \mnras, 424, 3123
  
%Cold dark matter haloes in the Planck era: evolution of structural parameters for Einasto and NFW profiles
\bibitem[Dutton \& Macci{\`o}(2014)]{Dutton14} Dutton,
  A.~A., \& Macci{\`o}, A.~V.\ 2014, \mnras, 441, 3359 


%% EEEEEEEEE

    
%% FFFFFFFFF

% The Cosmic Baryon Budget
\bibitem[Fukugita et al.(1998)]{Fukugita98} Fukugita, M., Hogan, C.~J., Peebles, P.~J.~F.\ 1998, \apj, 503, 518

%% GGGGGGGGG


%% HHHHHHHHH

% Large-scale mass distribution in the Illustris simulation
\bibitem[Haider et al.(2016)]{Haider16}
Haider, M., Steinhauser, D., Vogelsberger, M., et al.\ 2016, \mnras, 457, 3024

% Distributions of the Baryon Fraction on Large Scales in the Universe
\bibitem[He et al.(2005)]{He05}
He, P., Feng, L.~L., Fang, L.~Z.\ 2005, \apj, 623, 601

%% JJJJJJJJJJJ


%% KKKKKKKKKKK

%A superbubble feedback model for galaxy simulations
\bibitem[Keller et al.(2014)]{Keller14} Keller, B.~W., Wadsley, 
  J., Benincasa, S.~M., \& Couchman, H.~M.~P.\ 2014, \mnras, 442, 3013

% Stellar mass -- halo mass relation and star formation efficiency in
% high-mass halos
\bibitem[Kravtsov et al.(2014)]{Kravtsov14} Kravtsov, A., 
Vikhlinin, A., \& Meshscheryakov, A.\ 2014, arXiv:1401.7329 

  
%% LLLLLLLLL


%% MMMMMMMMM

% On the origin of the warm-hot absorbers in the Milky Way's halo
\bibitem[Marasco et al.(2013)]{Marasco13} 
Marasco, A., Marinacci, F., Fraternali, F.\ 2013, \mnras, 433, 1634

% The Baryon Content of Cosmic Structures
\bibitem[McGaugh et al.(2010)]{McGaugh10} McGaugh, S.~S., 
Schombert, J.~M., de Blok, W.~J.~G., Zagursky, M.~J.\ 2010, \mnras,
708, 14

% Galactic star formation and accretion histories from matching galaxies to dark matter haloes
\bibitem[Moster et al.(2013)]{Moster13} Moster, B.~P., Naab, T., 
\& White, S.~D.~M.\ 2013, \mnras, 428, 3121 


%% NNNNNNNNN


%% OOOOOOOO


%% PPPPPPPPP

%Planck 2013 results. XVI. Cosmological parameters
\bibitem[the Planck Collaboration et  al.(2014)]{Planck14}
  Planck Collaboration, Ade, P.~A.~R., Aghanim, N., et al.\ 2014,
  \aap, 571, AA16 
 
% A Budget and Accounting of Metals at z ~ 0: Results from the COS-Halos Survey
\bibitem[Peeples et al.(2014)]{Peeples14} Peeples, M.~S., Werk, J.~K., Tumlinson, J., et al.\ 2014, \apj, 786, 54

% The baryon content of the universe
\bibitem[Persic \& Salucci(1992)]{Persic92} Persic, M., Salucci, P.\ 1992, \mnras, 258, 14

%pynbody: N-Body/SPH analysis for python
\bibitem[Pontzen et al.(2013)]{Pontzen13} Pontzen, A., Ro{\v s}kar, R., Stinson, G., \& Woods, R.\ 2013, Astrophysics Source Code Library, 1305.002 


%% RRRRRRRRRRR


%% SSSSSSSSSSSS

% The enrichment of the intergalactic medium with adiabatic feedback - I. Metal cooling and metal diffusion
\bibitem[Shen et al.(2010)]{Shen10} Shen, S., Wadsley, J., 
\& Stinson, G.\ 2010, \mnras, 407, 1581 

% On the structure of hot gas in haloes: implications for the LX-TX relation and missing baryons
\bibitem[Sharma et al.(2012)]{Sharma12}
Sharma, P., McCourt, M., Parrish, I.~J., Quataert, E.\ 2012, \mnras, 427, 1219

% The Baryon Census in a Multiphase Intergalactic Medium: 30% of the Baryons May Still be Missing
\bibitem[Shull et al.(2012)]{Shull12}
Shull, J.~M., Smith, B.~D., Danforth, C.~W.\ 2012, \apj, 759, 23

% Diffuse Coronae in Cosmological Simulations of Milky Way-sized Galaxies
\bibitem[Sokolowska et al.(2016)]{Sokolowska16}
Sokolowska, A., Mayer, L., Babul, A., Madau, P., Shen, S.\ 2016, \apj, 819, 21

% A Universal Density Structure for Circum-Galactic Gas
\bibitem[Stern et al.(2016)]{Stern16} 
Stern, J., Hennawi, J.~F., Prochaska, J.~X., Werk, J.~K.\ 2016, arXiv:1604.02168

%Making Galaxies In a Cosmological Context: the need for early stellar feedback
\bibitem[Stinson et al.(2013)]{Stinson13} Stinson, G.~S., Brook, 
C., Macci{\`o}, A.~V., et al.\ 2013, \mnras, 428, 129 

% NIHAO III: the constant disc gas mass conspiracy
\bibitem[Stinson et al.(2015)]{Stinson15} Stinson, G.~S., Dutton, A.~A., Wang, L., et al.\ 2015, \mnras, 454, 1105 


%% TTTTTTTTTTT

% Not Dead Yet: Cool Circumgalactic Gas in the Halos of Early-type Galaxies
\bibitem[Thom et al.(2012)]{Thom12}
Thom, C., Tumlinson, J., Werk, J.~K.\ 2012, \apjl, 758, L41

% The Large, Oxygen-Rich Halos of Star-Forming Galaxies Are a Major Reservoir
% of Galactic Metals
\bibitem[Tumlinson et al.(2011)]{Tumlinson11} Tumlinson, J., Thom, C., Werk, J., et al.\ 2011, Science, 334, 948

% The COS-Halos Survey: Rationale, Design and a Census of Circumgalactic Neutral Hydrogen
\bibitem[Tumlinson et al.(2013)]{Tumlinson13} Tumlinson, J., Thom, C., Werk, J., et al.\ 2013, \apj, 777, 59

%% VVVVVVVVVVVV



%% WWWWWWWWWWWWWW

%Gasoline: a flexible, parallel implementation of TreeSPH
\bibitem[Wadsley et al.(2004)]{Wadsley04} Wadsley, J.~W., Stadel, 
J., \& Quinn, T.\ 2004, \na, 9, 137 

%On the treatment of entropy mixing in numerical cosmology
\bibitem[Wadsley et al.(2008)]{Wadsley08} Wadsley, J.~W., 
Veeravalli, G., \& Couchman, H.~M.~P.\ 2008, \mnras, 387, 427 

  %NIHAO project I: Reproducing the inefficiency of galaxy formation across cosmic time with a large sample of cosmological hydrodynamical simulations
\bibitem[Wang et al.(2015)]{Wang15} Wang, L., Dutton, A.~A.,  Stinson, G.~S., et al.\ 2015, \mnras, 454, 83
  
% The COS-Halos Survey: Keck LRIS and Magellan MagE Optical Spectroscopy
\bibitem[Werk et al.(2012)]{Werk12} Werk, J.~k., Prochaska, J.~X., Thom, C., et al.\ 2012, \apjs, 198, 3

% The COS-Halos Survey: An Empirical Description of Metal-line Absorption in the Low-redshift Circumgalactic Medium
\bibitem[Werk et al.(2013)]{Werk13} Werk, J.~k., Prochaska, J.~X., Thom, C., et al.\ 2013, \apjs, 204, 17

%The COS-Halos Survey: Physical Conditions and Baryonic Mass in the Low-redshift Circumgalactic Medium
\bibitem[Werk et al.(2014)]{Werk14} Werk, J.~k., Prochaska, J.~X., Thom, C., et al.\ 2014, \apj, 792, 8


%% YYYYYYYYYYYY

% The Temperature Structure of the Warm-Hot Intergalactic Medium
\bibitem[Yoshida et al.(2005)]{Yoshida05} 
Yoshida, N., Furlanetto, S.~R., Hernquist, L.\ 2005, \apj, 618L, 91

%% ZZZZZZZZZZZZZ

% Dynamical effect of the turbulence of the intergalactic medium on the baryon fraction distribution
\bibitem[Zhu et al.(2011)]{Zhu11} Zhu, W., Feng, L.~L., Fang, L.~Z.\ 2011, \mnras, 415, 1093

\end{thebibliography}

\label{lastpage}
\end{document}
